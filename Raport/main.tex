\documentclass{article}

\usepackage[polish]{babel}
\usepackage[utf8]{inputenc}
\usepackage[T1]{fontenc}

\usepackage{graphicx}
\usepackage[colorlinks=true, linkcolor=blue, urlcolor=blue, citecolor=gray]{hyperref}
\usepackage{url}

\usepackage{xcolor}    % <-- potrzebne do \definecolor
\usepackage{listings}  % <-- potrzebne do \lstset

\usepackage[backend=biber,style=numeric]{biblatex}
\addbibresource{bibliografia.bib}

\title{Symulacja procesów powstawania gatunków w ekosystemie w efekcie występowania barier geograficznych}
\author{Izabela Bubula, Andrzej Kotulski}
\date{Listopad 2025}

% Kolory inspirowane IDE
\definecolor{codebg}{RGB}{245,245,245}
\definecolor{codecomment}{RGB}{106,153,85}
\definecolor{codekeyword}{RGB}{0,0,205}
\definecolor{codestring}{RGB}{163,21,21}
\definecolor{codenumber}{RGB}{128,128,128}
\definecolor{codefunc}{RGB}{43,145,175}

\lstset{
  language=Python,
  backgroundcolor=\color{codebg},
  basicstyle=\ttfamily\small,
  keywordstyle=\color{codekeyword}\bfseries,
  commentstyle=\color{codecomment}\itshape,
  stringstyle=\color{codestring},
  numberstyle=\tiny\color{codenumber},
  numbers=left,
  stepnumber=1,
  numbersep=8pt,
  breaklines=true,
  showstringspaces=false,
  tabsize=2,
  frame=none,
  captionpos=b,
  emph={fitness,mutate,get_neighbors,can_move},
  emphstyle=\color{codefunc}\bfseries
}

\begin{document}
\maketitle


\section{Wprowadzenie}

Gatunki i ich różnorodność, którą można zaobserwować w odmiennych środowiskach, jest efektem procesów ewolucyjnych które zachodzą w długim czasie. Różnorodność ta spowodowana jest barierami przepływu genów między osobnikami. Jednym z bardziej znaczących mechanizmów powstawania nowych gatunków są bariery geograficzne. które poprzez podział populacji tworzą izolowane grupy. Bariery geograficzne można podzielić na 2 grupy:

\begin{itemize}
    \item Fizyczne:
         \begin{itemize}
            \item pasma górskie,
            \item jeziora,
            \item rzeki,
            \item pustynie,
            \item wyspy,
            \item obszary o skrajnie odmiennych warunkach
      \end{itemize}
    \item Środowiskowe:
        \begin{itemize}
            \item strefy klimatyczne,
            \item zróżnicowanie warunków życia
        \end{itemize}
\end{itemize}

Powstawianie lub zanikanie tych barier jest ściśle związane z dynamiką klimatu oraz zmian terenowych związanych m.in. z użytkowaniem terenu. Dzięki rozwojowi modelowania komputerowego możemy coraz dokładniej śledzić sposób w jaki izolacja przestrzenna wpływa na zróżnicowanie genetyczne.

Cel tego projektu to zaprojektowanie i implementacja modelu symulacyjnego, który będzie odwzorowywał proces zmian i powstawania gatunków w ekosystemie, w którym występują bariery geograficzne. Dzięki zaprojektowanemu modelowi będziemy mogli analizować:
\begin{itemize}
    \item wielkości i rozmieszczenie barier 
    \item parametry demograficzne (np. liczebności, migracje)
    \item tempo i charakter środowiska oraz jego zmiany
\end{itemize}
oraz ich wpływ na izolacje genetyczne, różnice populacyjne oraz stabilność powstałych genów. 

\section{Przegląd aktualnego stanu badań}
\subsection{Rola barier geograficznych i klimatycznych w powstawaniu nowych gatunków}

Powstawanie nowych gatunków bądź nasilanie różnic genetycznych może być wzmocnione poprzez przedstawienie barier geograficznych. Jest to jedno z klasycznych założeń powstawania nowych gatunków. W najnowszych pracach przeczytać można iż nie tylko fizyczne granice mogą być uważane za bariery geograficzne ale również bariery środowiskowe bądź klimatyczne spełniają podobne role.  

W artykule \textit{Konserwatyzm niszy ekologicznej stymuluje dywersyfikację w odpowiedzi na zmiany klimatyczne} (ang. Ecological niche conservatism spurs diversification in re-
sponse to climate change)\cite{1} można przeczytać, że mała zdolność gatunków do adaptacji w innych warunkach klimatycznych (konserwatyzm niszy klimatycznej) ma związek z tempem różnicowania linii ewolucyjnych. "Strefy nieodpowiednie" tworzone przez różnice klimatyczne, które izolują dawniej ciągłe populacje, przerywają przepływ genów sprzyjając jednocześnie do powstawania gatunków żyjących w odrębnych miejscach. Jeżeli gatunki nie są w stanie przystosować się wystarczająco szybko do warunków pośrednich to bariery te mogą utrzymywać się przez długi czas.

Podobne wnioski można znaleźć w artykule \textit{Schronienie klimatyczne i izolacja geograficzna przyczyniają się do specjacji i rozbieżności genetycznej u piwonii drzewiastych Himalajów i Hengduan} (ang. Climatic Refugia and Geographical Isolation Contribute to the Speciation and Genetic Divergence in Himalayan-Hengduan Tree Peonies (Paeonia delavayi and Paeonia ludlowii). Badane tam były dwa blisko spokrewnione gatunki drzewiastych piwonii, które mają wspólny historyczny genotyp, ale zupełnie inne rozmieszczenie geograficzne, P. delavayi można znaleźć w wielu prowincjach zachodnio-północnych Chin, gdzie klimat jest łagodniejszy, a P. ludlowii występuje tylko w małym obszarze Tybetu ze znacznie chłodniejszym i suchszym klimatem. P. ludlowii wykazała się małą ilością haplotypów co świadczy o małej różnorodności wewnątrzgatunkowej jak i silnym dryfem genetycznym. W próbkach od P. delavayi różnorodność haplotypów była o wiele większa co pokazuje długotrwałą oddzielną ewolucję tego podgatunku.\cite{2}

\subsection{Modele symulacyjne specjacji w przestrzennie złożonych ekosystemach}
Modele symulacyjne są ważną częścią badań, dzięki którym można zobaczyć efekty interakcji osobników, mutacji czy doboru naturalnego w znacznie szybszym czasie niż oczekiwanie na rezultaty badań fizycznych.

W pracy Specjacja z przepływem genów w heterogenicznym wirtualnym świecie (ang. Speciation with gene flow in a heterogeneous virtual world) autorzy badali czy drobne przeszkody fizyczne oraz niejednorodne środowisko przestrzenne będą mieć wpływ na proces powstawania nowych gatunków przy utrzymującym się przepływie genów. Z symulacji stwierdzono że obecność małych przeszkód fizycznych, nawet przy przepływie genów przyspiesza czas różnicowania się populacji i może prowadzić do powstawania wyraźnych grup genetycznych. Wnioski z tej symulacji są takie, że powstawanie nowych genów nie wymaga całkowitej izolacji genetycznej. Małe bariery oraz zmienność środowiska są wystarczającymi czynnikami. \cite{3}

Wpływ zmian środowiskowych na grupy o ograniczonym przepływie genów zaprezentowali Osmar Freitas, Sabrina B.L. Araujo oraz Paulo R.A. Campos w swojej pracy \textit{Specjacja w modelu metapopulacji po zmianach środowiskowych} (ang. Speciation in a metapopulation model upon environmental changes). Model badawczy składał się z podzielonego na wiele części siedliskowych ekosystemu. Grupy te połączone były ruchem osobników, który zmieniał się pod wpływem zmian klimatycznych. Zaobserwowano, że ruchy te zanikały i powstawały, co przekładało się na zmiany w strukturach gatunkowych. Wyniki pokazały że rozpad gatunków na linie o ograniczonym przepływie genów mogą być spowodowane zmniejszoną tendencją do przemieszczania się osobników, które zaadaptowały się do lokalnych warunków,  przy wystarczająco zróżnicowanym środowisku.\cite{4}

\subsection{Genomowy krajobraz specjacji i identyfikacja barier przepływu genów}

W badaniach nad specjacją coraz większą rolę odgrywa analiza genomu, która pozwala wykrywać fragmenty odpowiadające za ograniczony przepływ genów między populacjami. Jednak wzory zróżnicowania genomowego – \textit{tzw. „wyspy różnicowania”} \cite{2} – mogą wynikać nie tylko z barier przepływu genów, ale także z innych procesów, takich jak selekcja tła czy różnice w rekombinacji.

Dostępne narzędzia analityczne, m.in. metoda \textit{gIMble} \cite{6}, umożliwiają lokalne oszacowanie natężenia przepływu genów i efektywnego rozmiaru populacji oraz identyfikację regionów genomu związanych z ograniczoną migracją.

Najnowsze przeglądy podkreślają, że \textit{lokalna adaptacja} może inicjować specjację jeszcze zanim dojdzie do pełnej izolacji reprodukcyjnej \cite{7}.

W kontekście modelu symulacyjnego oznacza to konieczność monitorowania takich parametrów jak natężenie przepływu genów, siła selekcji lokalnej czy struktura bariery, aby móc porównywać otrzymane wyniki z danymi empirycznymi.

\subsection{Zmiany klimatu i dynamiczne bariery w procesie specjacji}

Zmiany klimatu mogą tworzyć \textit{„miękkie” bariery}, takie jak niekorzystne strefy klimatyczne, które utrudniają kontakt między populacjami. Może to prowadzić do powstawania klimatycznych korytarzy i blokad, wpływających na rozmieszczenie gatunków i zwiększających izolację \cite{1}.

Badania nad refugialnymi populacjami roślin pokazują, że oddzielenie takich ostoi przez bariery klimatyczne lub terenowe sprzyja długotrwałemu zróżnicowaniu genetycznemu \cite{2}.

W modelach symulacyjnych warto więc uwzględnić zmienność barier w czasie – ich pojawianie się, zanikanie oraz zmiany kosztu migracji – aby realistycznie odzwierciedlić procesy specjacyjne zachodzące w środowisku zmieniającym się pod wpływem klimatu.

\subsection{Wnioski dla projektowanego modelu}

Przegląd badań wskazuje, że \textit{struktura przestrzenna} ma kluczowe znaczenie dla procesu specjacji: zarówno fizyczne bariery, jak i nieciągłości klimatyczne wpływają na przepływ genów między populacjami \cite{3}.

Modele ekologiczne podkreślają rolę \textit{lokalnej adaptacji} oraz heterogeniczności środowiska, dlatego warto uwzględnić zmienne środowiskowe przypisane do poszczególnych obszarów symulacji \cite{7}.

Symulacje metapopulacyjne pokazują, że zmiany środowiskowe, wielkość populacji i struktura migracji silnie wpływają na przebieg specjacji \cite{4}.

Z kolei modele oparte na indywidualnych osobnikach wskazują, że nawet niewielkie \textit{bariery przestrzenne} mogą znacząco przyspieszać różnicowanie genetyczne \cite{3}.

Uwzględnienie dynamicznych barier, procesów lokalnej adaptacji oraz parametrów migracji pozwoli tworzyć realistyczne scenariusze prowadzące do powstawania gatunków.


\section{Opis rozwiązywanego problemu}

Celem projektu jest zbadanie, w jakim stopniu bariery geograficzne wpływają na rozdzielenie populacji i powstawanie odrębnych linii genetycznych (specjacja alopatryczna). Zagadnienie rozpatrywane jest przez modelowanie prostego ekosystemu dyskretnego, gdzie osobniki rozmieszczone są na siatce, poruszają się, rozmnażają i mutują. Kluczowe pytania to:
\begin{itemize}
  \item Jak obecność bariery (jej typ i pozycja) wpływa na tempo dywergencji genetycznej?
  \item Jakie znaczenie mają parametry demograficzne (rozmiar populacji, szybkość mutacji, intensywność migracji) dla procesu różnicowania?
  \item Które proste miary (np. odległość genotypów) dobrze odzwierciedlają powstawanie odrębnych skupisk genetycznych?
\end{itemize}


\section{Opracowany model symulacyjny}

Model został celowo uproszczony tak aby można było zrobić izolację wpływów barier przestrzennych. Najważniejsze elementy:

\subsection{Reprezentacja osobników}
Każdy osobnik to struktura zawierająca:
\begin{itemize}
  \item genotyp — wektor 8 wartości w przedziale $[0,1]$ (haploidalny),
  \item pozycję na siatce (indeks komórki),
  \item wiek (liczba przebytych generacji).
\end{itemize}

Średnia wartości wektora genotypu traktowana jest jako przybliżona cecha adaptacyjna osobnika.

\subsection{Środowisko i bariery}
Przestrzeń modelu to kwadratowa siatka $G$ (domyślnie $10\times10$). Każda komórka ma wartość środowiskową $e(x,y)\in[0,1]$. Bariery reprezentowane są jako macierz binarna — komórki z wartością 1 blokują migrację. Model obsługuje bariery pionowe, poziome i diagonalne.

\subsection{Procesy modelowane w każdej generacji}
Dla każdej generacji wykonywane są następujące kroki:
\begin{enumerate}
  \item \textbf{Ocena fitness}: W zaproponowanym modelu ewolucyjnym dopasowanie osobnika do środowiska (\textit{fitness}) jest określane na podstawie różnicy pomiędzy jego genotypem a lokalnymi warunkami środowiskowymi. Każdy osobnik posiada genotyp reprezentowany jako wektor wartości liczbowych, natomiast środowisko opisane jest pojedynczym parametrem środowiskowym.

Funkcję dopasowania można zapisać w postaci równania:

\begin{equation}
f = \frac{1}{1 + \left| \sum_{i=1}^{n} g_i - 2E \right|}
\end{equation}

gdzie:
\begin{itemize}
    \item $f$ -- wartość funkcji dopasowania (fitness),
    \item $g_i$ -- $i$-ty allel genotypu osobnika,
    \item $\sum_{i=1}^{n} g_i$ -- suma alleli genotypu,
    \item $E$ -- wartość środowiska w danej komórce siatki,
    \item $2E$ -- docelowa wartość genotypu optymalnie dopasowanego do środowiska.
\end{itemize}

Funkcja ta osiąga maksymalną wartość równą $1$ w sytuacji, gdy suma genotypu osobnika jest równa dwukrotności wartości środowiska. Wraz ze wzrostem różnicy pomiędzy genotypem a warunkami środowiskowymi wartość funkcji dopasowania maleje w sposób hiperboliczny. Zastosowanie wartości bezwzględnej powoduje, że zarówno nadmiar, jak i niedobór cechy genetycznej względem środowiska skutkują obniżeniem dopasowania.
  \item \textbf{Selekcja i reprodukcja}:  dla każdego osobnika obliczane jest jego dopasowanie do lokalnych warunków środowiskowych $fit = fitness(ind, e(x,y))$. Na tej podstawie wyznaczane jest prawdopodobieństwo rozrodu
  \[
  p_{\text{repro}} = p_{\text{base}} \cdot fit.
  \]
  Jeśli losowa zmienna o rozkładzie jednostajnym spełnia warunek $U(0,1) < p_{\text{repro}}$, osobnik generuje jednego potomka w tej samej komórce siatki.
  \item \textbf{Mutacja}: genotyp potomka powstaje w wyniku zastosowania funkcji mutacji do genotypu rodzica. Każdy allel mutuje z prawdopodobieństwem $\mu = p_{\text{mutation}}$, zgodnie z parametrami symulacji. Wartości genotypu po mutacji są ograniczane do dopuszczalnego przedziału (np. $[0,1]$).

  \item \textbf{Migracja}:  każdy osobnik może przemieścić się do jednej z komórek sąsiednich z prawdopodobieństwem $p_{\text{mig}} = p_{\text{migration}}$. Ruch jest wykonywany wyłącznie wtedy, gdy istnieje dostępna komórka sąsiednia oraz gdy bariera przestrzenna nie blokuje przejścia pomiędzy komórką początkową i docelową.

  \item \textbf{Regulacja liczebności}:  po dodaniu potomków populacja grupowana jest według komórek siatki. Jeżeli liczba osobników w danej komórce przekracza maksymalną pojemność $K = \text{max\_per\_cell}$, nadmiar osobników jest usuwany losowo, tak aby zachować nie więcej niż $K$ osobników w każdej komórce.

  \item \textbf{Grupowanie}:  w przedstawionej implementacji pojedynczego kroku symulacji etap grupowania osobników w klastry nie jest realizowany bezpośrednio. Analiza skupisk genetycznych wykonywana jest w osobnym etapie przetwarzania wyników, poza funkcją pojedynczego kroku symulacji.

\end{enumerate}

\subsection{Metryki oceny}
Do oceny stopnia dywergencji i różnorodności używane są proste, interpretowalne miary:
\begin{itemize}
  \item odchylenie standardowe wartości alleli w populacji — miara różnorodności wewnątrz populacji,
  \item odległość euklidesowa między średnimi genotypami dwóch skupisk (znormalizowana) — miara dywergencji między populacjami,
  \item liczba przestrzennych klastrów — przybliżenie liczby inkipientnych gatunków.
\end{itemize}

\section{Wykorzystane algorytmy i biblioteki}

\subsection{Algorytmy}
\begin{itemize}
  \item \textbf{Selekcja proporcjonalna do fitness}: losowanie rodziców z wagami równymi wartościom fitness.
  \item \textbf{Mutacja gaussowska}: niezależne, losowe perturbacje alleli z ograniczeniem do przedziału $[0,1]$.
  \item \textbf{Prosty algorytm grupowania (clustering)}: przegląd par osobników i łączenie tych w odległości poniżej progu (algorytm złożoności $O(N^2)$ — wystarczający dla niewielkich N).
  \item \textbf{Migracja lokalna}: ruch do sąsiednich komórek z kontrolą przez macierz barier.
\end{itemize}

\subsection{Algorytmy (przykłady z implementacji)}

\subsubsection{Reprodukcja zależna od fitness (zamiast selekcji proporcjonalnej)}
W implementacji nie zastosowano klasycznej selekcji ruletkowej (losowania rodziców z wagami równymi wartościom fitness).
Zamiast tego każdy osobnik w danym kroku symulacji niezależnie podejmuje próbę rozrodu, gdzie prawdopodobieństwo reprodukcji
jest proporcjonalne do dopasowania osobnika do środowiska:
\[
p_{\mathrm{repro}} = p_{\mathrm{base\_repro}} \cdot fitness.
\]
Potomek powstaje, gdy zachodzi warunek $U(0,1) < p_{\mathrm{repro}}$.

\begin{lstlisting}[language=Python, caption={Reprodukcja zależna od fitness w kroku symulacji}, label={lst:reproduction}]
env_val = env[ind.y, ind.x]
fit = fitness(ind, env_val)
p_repro = config.p_base_repro * fit

if random.random() < p_repro:
    child_genotype = mutate(ind.genotype, config.p_mutation)
    offspring.append(Individual(x=ind.x, y=ind.y,
                               genotype=child_genotype,
                               birth_time=current_time))
\end{lstlisting}

\subsubsection{Mutacja losowa alleli (zamiast mutacji gaussowskiej)}
Mutacja realizowana jest poprzez niezależne sprawdzenie każdego genu. Dla genu zachodzi mutacja z prawdopodobieństwem
$\mu = p_{\mathrm{mutation}}$. W przypadku mutacji wartość allelu jest losowana ponownie ze zbioru $\{0,1,2\}$.
Jest to mutacja dyskretna (nie gaussowska).

\begin{lstlisting}[language=Python, caption={Mutacja genotypu w funkcji \texttt{mutate}}, label={lst:mutation}]
def mutate(genotype: np.ndarray, p_mut: float):
    new_genotype = genotype.copy()
    for i in range(len(new_genotype)):
        if random.random() < p_mut:
            new_genotype[i] = random.randint(0, 2)
    return new_genotype
\end{lstlisting}

\subsubsection{Migracja lokalna (von Neumanna, 4-kierunkowa) z kontrolą bariery}
Migracja realizowana jest jako ruch do jednej z komórek sąsiednich (góra/dół/lewo/prawo) z prawdopodobieństwem
$p_{\mathrm{mig}} = p_{\mathrm{migration}}$. Ruch jest dozwolony wyłącznie wtedy, gdy komórka docelowa nie jest oznaczona jako bariera.

\begin{lstlisting}[language=Python, caption={Definicja sąsiedztwa von Neumanna}, label={lst:neighbors}]
def get_neighbors(x, y, width, height):
    neighbors = []
    if x > 0: neighbors.append((x - 1, y))
    if x < width - 1: neighbors.append((x + 1, y))
    if y > 0: neighbors.append((x, y - 1))
    if y < height - 1: neighbors.append((x, y + 1))
    return neighbors
\end{lstlisting}

\begin{lstlisting}[language=Python, caption={Sprawdzenie bariery i wykonanie migracji}, label={lst:migration}]
def can_move(x1, y1, x2, y2, barrier):
    return not barrier[y2, x2]

for ind in population:
    if random.random() < config.p_migration:
        neighbors = get_neighbors(ind.x, ind.y, width, height)
        if neighbors:
            new_x, new_y = random.choice(neighbors)
            if can_move(ind.x, ind.y, new_x, new_y, barrier):
                ind.x, ind.y = new_x, new_y
\end{lstlisting}

\subsubsection{Prosty algorytm grupowania (clustering) w oparciu o podobieństwo genetyczne}
W projekcie zastosowano prosty mechanizm klastrowania na podstawie odległości Hamminga pomiędzy genotypami.
Najpierw wyznaczana jest macierz odległości dla wszystkich par osobników (złożoność obliczeniowa $O(N^2)$),
a następnie osobniki o odległości mniejszej od zadanego progu są przypisywane do tego samego klastra.

\begin{lstlisting}[language=Python, caption={Klastrowanie osobników na podstawie odległości Hamminga}, label={lst:clustering}]
genotypes = np.array([ind.genotype for ind in population])
distances = squareform(pdist(genotypes, metric='hamming'))

clusters = []
assigned = set()

for i, ind in enumerate(population):
    if i in assigned:
        continue
    cluster = [ind]
    assigned.add(i)

    for j in range(i + 1, len(population)):
        if j in assigned:
            continue
        if distances[i, j] < threshold:
            cluster.append(population[j])
            assigned.add(j)

    clusters.append(cluster)
\end{lstlisting}


\subsection{Biblioteki i implementacja}
Model zaimplementowano w Pythonie. Kluczowe biblioteki:
\begin{itemize}
  \item \textbf{NumPy} — operacje macierzowe, generowanie losowych genotypów, szybkie obliczenia wektorowe;
  \item \textbf{Matplotlib} — rysowanie rozkładu przestrzennego populacji oraz wykresów czasowych;
  \item \textbf{SciPy} (moduł spatial.distance) — pomoc przy obliczaniu odległości genotypów.
\end{itemize}

Kod podzielono na moduły: rdzeń symulacji (funkcje inicjalizacji, krok symulacji, mutacje, migracja, grupowanie), skrypty uruchamiające eksperymenty oraz skrypty pomocnicze generujące wykresy i zapisujące wyniki.

\section{Złożoność obliczeniowa i uwagi praktyczne}

Każda iteracja symulacji wykonuje operacje proporcjonalne do liczby osobników $N$. Najbardziej kosztowny etap to grupowanie (parowe porównania), o złożoności $O(N^2)$. Dla typowych ustawień (kilkaset osobników) czas wykonania jest akceptowalny; przy większych N można rozważyć przyspieszenie (np. struktury przestrzenne — siatki, KD-tree).


\printbibliography

\section*{Załączniki}
\addcontentsline{toc}{section}{Załączniki}

Repozytorium projektu dostępne jest pod adresem:  
\url{https://github.com/akotu235/species-formation-simulation}

\end{document}
